% Options for packages loaded elsewhere
\PassOptionsToPackage{unicode}{hyperref}
\PassOptionsToPackage{hyphens}{url}
%
\documentclass[
]{article}
\usepackage{amsmath,amssymb}
\usepackage{lmodern}
\usepackage{iftex}
\ifPDFTeX
  \usepackage[T1]{fontenc}
  \usepackage[utf8]{inputenc}
  \usepackage{textcomp} % provide euro and other symbols
\else % if luatex or xetex
  \usepackage{unicode-math}
  \defaultfontfeatures{Scale=MatchLowercase}
  \defaultfontfeatures[\rmfamily]{Ligatures=TeX,Scale=1}
\fi
% Use upquote if available, for straight quotes in verbatim environments
\IfFileExists{upquote.sty}{\usepackage{upquote}}{}
\IfFileExists{microtype.sty}{% use microtype if available
  \usepackage[]{microtype}
  \UseMicrotypeSet[protrusion]{basicmath} % disable protrusion for tt fonts
}{}
\makeatletter
\@ifundefined{KOMAClassName}{% if non-KOMA class
  \IfFileExists{parskip.sty}{%
    \usepackage{parskip}
  }{% else
    \setlength{\parindent}{0pt}
    \setlength{\parskip}{6pt plus 2pt minus 1pt}}
}{% if KOMA class
  \KOMAoptions{parskip=half}}
\makeatother
\usepackage{xcolor}
\usepackage[margin=1in]{geometry}
\usepackage{graphicx}
\makeatletter
\def\maxwidth{\ifdim\Gin@nat@width>\linewidth\linewidth\else\Gin@nat@width\fi}
\def\maxheight{\ifdim\Gin@nat@height>\textheight\textheight\else\Gin@nat@height\fi}
\makeatother
% Scale images if necessary, so that they will not overflow the page
% margins by default, and it is still possible to overwrite the defaults
% using explicit options in \includegraphics[width, height, ...]{}
\setkeys{Gin}{width=\maxwidth,height=\maxheight,keepaspectratio}
% Set default figure placement to htbp
\makeatletter
\def\fps@figure{htbp}
\makeatother
\setlength{\emergencystretch}{3em} % prevent overfull lines
\providecommand{\tightlist}{%
  \setlength{\itemsep}{0pt}\setlength{\parskip}{0pt}}
\setcounter{secnumdepth}{-\maxdimen} % remove section numbering
\ifLuaTeX
  \usepackage{selnolig}  % disable illegal ligatures
\fi
\IfFileExists{bookmark.sty}{\usepackage{bookmark}}{\usepackage{hyperref}}
\IfFileExists{xurl.sty}{\usepackage{xurl}}{} % add URL line breaks if available
\urlstyle{same} % disable monospaced font for URLs
\hypersetup{
  pdftitle={Untitled},
  pdfauthor={Jamie Ash},
  hidelinks,
  pdfcreator={LaTeX via pandoc}}

\title{Untitled}
\author{Jamie Ash}
\date{2022-09-01}

\begin{document}
\maketitle

\textbf{Question 1.} Complete the following definitions:

\textbf{Definition 1.} An integer n is even if it can be divided by two
and the result is an intiger.

\[
\frac{n}{2} = i
\]

Where n is some number, and i is an intiger.

\textbf{Definition 2.} An integer m is odd if we can add one then divide
by two and the result is an intiger.

\[
\frac{(n+1)}{2} = i
\] Where n is some number, and i is an intiger.

\textbf{Question 2.} Use your definitions to prove the following
statement.

\textbf{Proposition 3.} If n is an even integer then \(n^2\) is an even
integer.

If n is an even integer then \(n = 2i\), where i is some intiger. Then
n\^{}2 is an even integer if \(n^2/2 = i\).

When we substitude 2i for n in \(n^2/2 = i\) we get the folowing.

\[
\frac{(2i)^2}{2} = i \tag{1}
\] We rearange the equation to be\ldots{}

\[
\frac{1}{2}(2i)(2i) \tag{2}
\] Then if we factor out the 2 we get\ldots{} \[
(i)(2i) \tag{3}
\] When two intigers are added or multiplied the result is an intiger.
So the result of \((i)(2i)\) is always an integer, and \(n^2\) is always
an even number.

This can be extended to \(n^x\) where \(x\) is some natural number. So
step 2 would become:

\$\$

i = \frac{1}{2}(2i)(2i)\ldots{} \textbackslash{} i = (i)(2i)\ldots{}
\$\$ Similarly \((i)(2i)...\) is entrely composed of intigers being
multiplied by one another, so the result is an intiger, and \(n^x\) is
even.

\end{document}
